\section{Introdução}
\subsection{Motivação}

\frame
{
	\frametitle{Motivação}
	
	The beamer class is a LaTeX class that allows you to create a beamer presentation. It can also be used to create slides. It behaves similarly to other packages like Prosper, but has the advantage that it works together directly with pdflatex, but also with dvips.
	Text describing the class in this presentation is taken from their website: \href{http://latex-beamer.sourceforge.net/}{http://latex-beamer.sourceforge.net/}.
}

\subsection{Objetivos}
\frame
{
	\frametitle{Objetivos}

	Once you have installed the beamer class~\footnote{It is already installed in my Fedora Core 4}, the basic steps to create a beamer presentation are the following:
	
	\begin{enumerate}
	\item Specify beamer as document class instead of article.
	\item Structure your LaTeX text using section and subsection commands.
	\item Place the text of the individual slides inside \\frame commands.
	\item Run pdflatex on the text (or latex and dvips).
	\end{enumerate}
}

\subsection{Justificativa}
\frame
{
	\frametitle{Justificativa}

	The beamer class has several useful features:
	
	\begin{itemize}
	\item You don't need any external programs to use it other than pdflatex, but it works also with dvips.
	\item You can easily and intuitively create sophisticated overlays.
	\item Finally, you can easily change the whole slide theme or only parts of it.
	\end{itemize}
}
